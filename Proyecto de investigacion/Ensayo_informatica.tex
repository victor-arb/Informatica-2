
\documentclass[12pt, letter]{article}
\usepackage[utf8]{inputenc}
\usepackage[spanish]{babel}
%\usepackage{times},puede ser arial 
\usepackage{csquotes}
\usepackage[left=2.54cm, right=2.54cm,top=2.54cm,bottom=2.54cm]{geometry}
\renewcommand{\baselinestretch}{1.5}
\usepackage[backend=biber,style=apa]{biblatex}
\bibliography{Referencias.bib}

\title{\huge{El nacimiento de la computación moderna}}
\author{Victor Manuel Arbeláez Ramírez \\ Facultad de ingeniería \\ Universidad de Antioquia}
\date{}

\begin{document}\raggedright

\maketitle

\section*{Resumen}

En este artículo se enunciarán cuáles fueron los pasos y procesos que se dieron en la humanidad a partir del siglo XIX para llegar al diseño de la computación que tenemos ahora. Para eso se realizó una investigación en diferentes fuentes en las que luego de hacer una exhaustiva selección se pudieron determinar cuáles de ellas eran las más óptimas para esta investigación.

\section*{Introducción}
La computación, como la conocemos hoy en día, es uno de los recursos más usados por las personas en sus diferentes labores y disciplinas. Esta herramienta es de gran importancia y utilidad para la elaboración de todo cuanto existe y aunque parece sencillo lleva tras de sí un largo proceso de investigaciones y teorías que a lo largo del tiempo fueron de gran controversia unas con otras (la crisis de los fundamentos) pero después de varios debates y descubrimientos se pudieron forjar las bases para algo tan maravilloso como lo es la computación moderna.

\section*{¿Cómo derivó la crisis de los fundamentos en la computación moderna?}

\setlength{\parindent}{31pt}
Después de la ya establecida revolución industrial, se empezó a poner el foco de atención en los campos de la ingeniería y la ciencia, por lo que era necesario apoyarse en las matemáticas que era la que los respaldaba.A inicios del siglo XX, empezaron a aparecer incongruencias en el manejo de las matemáticas, respecto al llamado infinito, las cuales terminaron por dividir a la comunidad científica en grupos que defendían diversos puntos de vista, llegando así a iniciar lo que posteriormente se conocería como la crisis de los fundamentos \parencite{computacionm}. Debido a esto empezaron a aparecer grandes protagonistas que empezarían a publicar sus trabajos para defender sus posturas y argumentar aquello que creían correcto, llevando así inicialmente a los matemáticos a controvertidos planteamientos que se irían resolviendo con el tiempo para dar lugar a los primeros pasos de la computación.

\setlength{\parindent}{31pt}
Inicialmente se encuentra el aporte de George Cantor, un matemático conjuntista quien fue el primero en abordar las temáticas del infinito, un tema al cual los científicos le rehuían ya que no había suficientes bases para trabajarlo. George Cantor estableció el concepto sobre el infinito y las teorías de conjuntos que indican que hay infinitos de diferentes tamaños, teorías que establecerían posteriormente las bases para las operaciones de la suma y la multiplicación \parencite{Cantor}. Aún así en su momento, esto no terminaba por convencer a la comunidad científica que en su época había sido dividida por la crisis de los fundamentos, y creían en gran parte que las matemáticas eran absolutas y completas; pero más tarde aparecerían otros dos grandes protagonistas que lograrían ampliar el razonamiento de Cantor y terminar con la creencia de la época.

\setlength{\parindent}{31pt}
No mucho tiempo después de las publicaciones hechas por Cantor, aparece David Hilbert, quien era un matemático de gran renombre en la comunidad científica, y había hecho grandes aportes a la geometría y otras ramas de la matemática, para después empezar a abordar la llamada problemática del infinito con el programa Hilbert, el cual sostenía que los sistemas axiomáticos cumplían con las propiedades de ser consistentes, finitos y completos, es decir que se sabían si eran falsos o verdaderos \parencite{Hilbert1, Hilbert2}. Hilbert quien era gran partidario de las matemáticas infalibles había estado a punto de publicar su obra que demostraba el programa Hilbert, pero al final fue refutado por Kurt Gödel, otro científico quien hizo una publicación en la que demostraba que las bases con respecto al infinito de Hilbert no se podían llevar a cabo.

\setlength{\parindent}{31pt}
Kurt Gödel fue un matemático que incluso antes de los 30 años ya estaba publicando los trabajos que lo distinguirían conocidos como los teoremas de la incompletitud, en los que expresaba que había problemas matemáticos de los cuales no podían demostrase su veracidad
\parencite{KurtG1,KurtG2}. Como se puede ver por el contexto, Gödel fue uno de los que con su lógica expandió lo expresado por Cantor, y refuto los axiomas de Hilbert ya que demostró que un sistema no puede ser finito y completo a la vez. Gracias a sus publicaciones, dio paso al protagonista que sería el punto de inflexión para lo que sería el comienzo de la computación y más exactamente lo que desencadenaría la computación moderna posteriormente, Alan Turing.

\setlength{\parindent}{31pt}
Uno de los aspectos por los que se recuerda a Alan Turing es por su aspecto criptográfico, siendo de gran importancia para la resolución del conflicto de la segunda guerra mundial ayudando a descifrar los códigos que transmitía la máquina enigma, un dispositivo que llevaba mensajes considerados “indescifrables” por muchos entre el ejército alemán, y que al final llegaría a poner a Europa en jaque \parencite{enigma}. Debido a que Turing fue decisivo en el desarme del sistema que transmitían los códigos enigma, a menudo se desconoce entre la mayoría de las personas su gran aporte a las matemáticas y en especial con relación a la problemática del infinito que había estado planteando controversia por varios años ya desde que se inició la crisis de los fundamentos. 
\setlength{\parindent}{31pt}
Alan Turing, fue un matemático e informático quien a temprana edad dio muestras de su genialidad publicando el artículo que traduce “el problema de decisión”, en el que definía lo que era y no era computable, expresando que aquello que era computable, se podría resolver mediante un algoritmo, lo demás no era resoluble; se basó en la máquina de Turing, un dispositivo imaginario que no logró materializar por la tecnología de la época, que permitía saber que algoritmos eran programables \parencite{Alanturing}.  Turing siguió y amplió el esquema que habían planteado los precursores a su teoría, Georg Cantor y Kurt Gödel, diciendo que no solo hay infinitos de distintos tamaños, que había sistemas matemáticos en los que no se podía verificar su veracidad, sino que además infiere que no se sabrían cuáles eran esos problemas hasta que se computasen, y se probara si el algoritmo funciona, lo que tiempo después terminaría siendo las bases del computador.

\setlength{\parindent}{31pt}
Después se entraría en una época en la que se desarrollaría la tecnología necesaria para aplicar los conocimientos teóricos que planteó Turing y así dar paso a los ordenadores y demás dispositivos informáticos que se utilizan hoy en día. Si observamos a cada uno de estos matemáticos con los aportes que le hicieron a la ciencia en todo su conjunto, podemos visibilizar que de las grandes crisis siempre se desprenden grandes soluciones. Es así pues como la crisis de los fundamentos iniciada en el siglo XX, termina por derivar en lo que se establece como la computación moderna.



\printbibliography[title={Referencias}]


\end{document}\raggedright

